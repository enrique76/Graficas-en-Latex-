\documentclass{article}
\usepackage[utf8]{inputenc}
\usepackage{tikz}
\usepackage{pgfplots}
\usepackage[spanish]{babel}
\usepackage{amsmath}
\usepackage{geometry}
\usepackage{pgf-pie}
\usetikzlibrary{patterns}
\usepackage{tikz-3dplot}
\usetikzlibrary{math}
%\usepackage[active,tightpage]{preview}
%\PreviewEnvironment{tikzpicture}
%\setlength\PreviewBorder{1pt}
\pgfplotsset{compat=1.12}
\usepgfplotslibrary{polar}

\title{Gráficas }

\geometry{top=1cm}

\definecolor{crema}{HTML}{FFE3B3} 
\definecolor{mostaza}{HTML}{FFC172} 
\definecolor{verde}{HTML}{63CAA7} 
\definecolor{azul}{HTML}{26798e}
\definecolor{rojo}{HTML}{DD4470}

\begin{document}
\maketitle
\tableofcontents
\section{Gráficas 2D}

\subsection{Polinomial}
%\eqref{gra:fx12d}
\begin{figure}[h!]
    \centering
    \begin{tikzpicture}[scale=1]
        \begin{axis}[xmin=-5,xmax=5,ymin=-5,ymax=5,axis  lines=center,
        xlabel=$x$, xlabel style={at={(ticklabel* cs:1.05)}},
        ylabel=$y$, ylabel style={at={(ticklabel* cs:1.05)}},
        grid]
            \addplot[red!80!white,thick,samples=200
            ,domain=-3:3]{x^4-0.5*x^3-2*x^2-x+1};
            \node at (axis cs:3,3){$F(x)$};
        \end{axis}
    \end{tikzpicture}
    \caption{Función $F(x) = x^4 - 0.5x^3 - 2x^2 - x +1$}
    \label{gra:fx12d}
\end{figure}

\centering
% \begin{tikzpicture}
% \draw[very thin,color=gray] (-0.1,-1.1) grid (7,1.1);
% \draw[->] (-0.2,0) -- (7.5,0) node[right] {$x$};
% \draw[->] (0,-1.2) -- (0,1.5) node[above] {$y$};
% \draw[blue,thick,->] plot[domain=0:7] (\x,{sin(\x r)}) 
% 	node[right] {$h(x) = \sin x$};
% \draw[red,thick,->] plot[domain=0:6.35] (\x,{cos(\x r)}) 
% 	node[above] {$g(x) = \cos x$};
% \end{tikzpicture}

\bigskip
\begin{tikzpicture}
\begin{axis}[
scale = 1.2,
xmin = 0, xmax = 10,
ymin = 0, ymax = 10,
axis lines* = left,
xtick = {0}, ytick = \empty,
axis on top,
clip = false,
]
% Supply and Demand Curves
\addplot[color=blue, very thick] coordinates {(1, 9) (9, 1)};
\addplot[color=red, very thick] coordinates{(3, 1) (6, 9)};

% Quantity and Price Lines
\addplot[color=black, dashed, thick] coordinates {(0, 5.36) (4.64, 5.36) (4.64, 0)};
\addplot[color=black, dashed, thick] coordinates {(0, 2) (10, 2)};
\addplot[color=black, dashed, thick] coordinates {(3.38, 2) (3.38, 0)};
\addplot[color=black, dashed, thick] coordinates {(8, 2) (8, 0)};

% Coordinate points
\addplot[color=black, mark=*, only marks, mark size=3pt] coordinates {(3.38, 2) (4.64, 5.36) (8,2)};

% Labels
\node [right] at (10, 2) {Ceiling};

\node [right] at (4.7, 5.4) {$E$};
\node [above left] at (3.38, 2) {$E^{\prime}$};

\node [above] at (current axis.above origin) {Rent, $P$};
\node [right] at (current axis.right of origin) {Apartments, $Q$};

\node [right] at (9, 1) {$D$};
\node [right] at (6, 9) {$S$};

\node [left] at (0, 5.36) {$P_E$};
\node [left] at (0, 2) {$P_c$};

\node [below] at (4.64, 0) {$Q_E$};
\node [below] at (3.38, 0) {$Q_s$};
\node [below] at (8, 0) {$Q_d$};

\draw[|-|] (3.38, -1) to (4.64, -1);
\node [below] at (4, -1) {Apartments lost};
\draw[|-|] (3.38, -2) to (8, -2);
\node [below] at (5.7, -2) {Shortage};
\end{axis}
\end{tikzpicture}

\bigskip
\begin{tikzpicture}
\begin{axis}[
scale = 1.2,
xmin = 0, xmax = 10.5,
ymin = 0.01, ymax = 10.5,
axis lines* = left,
xtick = {0,1,2,3,4,5,6,7,8,9,10},
ytick = {0,1,2,3,4,5,6,7,8,9,10},
grid = both,
minor tick num = 1,
grid style = solid,
minor grid style = dotted,
clip = false,
]
% Supply and demand curves
\addplot[color = blue, very thick] coordinates {(1, 9) (9, 1)};
\addplot[color = red, very thick] coordinates {(1, 1) (9, 9)};

% Dashed lines
\addplot[color = black, dashed, thick] coordinates {(0, 5) (5, 5) (5, 0)};

% Coordinate points
\addplot[color = black, mark = *, only marks, mark size = 3pt] coordinates {(5, 5)};

% Labels
\node [right] at (current axis.right of origin) {Widgets, $Q$};
\node [above] at (current axis.above origin) {Price, $P$};
\node [above = 5pt, fill = white] at (5, 5.2) {$E$};
\node [left = 10pt] at (0, 5) {$P_E$};
\node [below = 10pt] at (5, 0) {$Q_E$};
\node [right, fill = white] at (9, 1) {$D$};
\node [right, fill = white] at (9, 9) {$S$};
\end{axis}
\end{tikzpicture}

% \begin{tikzpicture}[scale=0.8]
% \draw[blue,thick,->] (-2.5,0)--(5,0) node[right,below] {$x$}; % Eje x
% % Enumeración del eje x
% \foreach \x/\xtext in {-2/-2, -1/-1, 1/1, 2/2, 3/3, 4/4} 
% \draw[shift={(\x,0)},blue] (0pt,2pt)--(0pt,-2pt) node[below] {$\xtext$};
% %
% % Enumeración del eje y
% \foreach \y/\ytext in {-1/-1, 1/1, 2/2, 3/3, 4/4} 
% \draw[shift={(0,\y)},blue] (2pt,0pt)--(-2pt,0pt) node[left] {$\ytext$};
% \draw[blue,thick,->] (0,-2)--(0,4.25) node[left,above] {$y$}; % Eje y
% %
% \node[below] at (-0.25,0){$O$};
% \draw[red,thick,<->] (-1,4) -- (4,-1); 
% \node[color=red,right] at (4,-1) {$x+y=3$};
% %
% \draw[red,thick,<->] (-1,-2) -- (4,3); 
% \node[color=red,right] at (4,3) {$x-y=1$};
% %
% \draw[fill=black] (2,1) circle (2pt);
% \end{tikzpicture}

\newpage
\bigskip
\subsection{Elipse}

\begin{figure}[h!]
    \centering
    \begin{tikzpicture}[scale=1]
        \begin{axis}[
        grid,
        xlabel = $x$,
        axis x line=center,
        ylabel = $y$,
        axis y line=center,
        xmin = 0,
        xmax = 8,
        ymin = 0,
        ymax = 4.5
        ]
        
        \addplot[
        domain=0:360,
        samples=100,
        thick,
        red,
        ]({3+3*cos(x)},{2+2*sin(x)});
        \end{axis}
    \end{tikzpicture}
    \caption{Elipse}
    \label{gra:elipse}
\end{figure}

\subsection{Hipérbola}
\begin{figure}[h!]
    \centering
    \begin{tikzpicture}[scale=1]
        \begin{axis}[
        grid,
        xlabel = $x$,
        axis x line=center,
        ylabel = $y$,
        axis y line=center,
        xmin = 0,
        xmax = 8,
        ymin = 0,
        ymax = 4.5
        ]
        
        \addplot[
        domain=-89:89,
        samples=100,
        thick,
        red,
        ]({3+1/cos(x)},{2+1*tan(x)});
        
        \addplot[
        domain=91:269,
        samples=100,
        thick,
        red,
        ]({3+1/cos(x)},{2+1*tan(x)});
        \end{axis}
    \end{tikzpicture}
    \caption{Hipérbola}
    \label{gra:hiperbola}
\end{figure}

\newpage

\subsection{Área bajo la Curva}
% \begin{figure}[h!]
%     \centering
%     \begin{tikzpicture}
%         \begin{axis}[
%         width=10cm,
%         height=6cm,
%         xmin=-1,
%         xmax=5,
%         ymin=0,
%         ymax=3.5,
%         xlabel=$x$,
%         ylabel=$y$,
%         xlabel style={at={(ticklabel* cs:1.05)}},
%         ylabel style={at={(ticklabel* cs:1.05)}},
%         axis lines=center
%         ]
%         \addplot[verde,thick,samples=200,domain=-1:5]{1+sqrt(x)};
%         \filldraw[fill=azul!20,draw=black] (0,0) -- plot [domain=-1:4] (\x,{1+sqrt(\x)}) -- (4,0) -- cycle;
%         \end{axis}
%     \end{tikzpicture}
%     \caption{Área bajo la Curva}
%     \label{gra:abc}
% \end{figure}

\subsection{Área entre Dos Curvas}
\begin{figure}[h!]
    \centering
    \begin{tikzpicture}
        \begin{axis}[
        width=10cm,
        height=6cm,
        xmin=-0.5,
        xmax=2.5,
        ymin=-0.3,
        ymax=1.2,
        xlabel=$x$,
        ylabel=$y$,
        xlabel style={at={(ticklabel* cs:1.05)}},
        ylabel style={at={(ticklabel* cs:1.05)}},
        axis lines=center
        ]
        
        \filldraw[draw=black,fill=cyan!20]
        plot [smooth,domain=0:1] (\x,{2*(\x) - (\x)^2}) --
        plot [smooth,domain=1:0] (\x,\x^4) -- cycle;
        
        \addplot[verde,thick,samples=200,domain=-0.2:2]{2*(\x) - (\x)^2};
        \addplot[verde,thick,samples=200,domain=-0.2:1.1]{(\x)^4};
        
        \end{axis}
    \end{tikzpicture}
    \caption{Área entre dos Curvas}
    \label{gra:abc}
\end{figure}

\bigskip
\begin{tikzpicture}
\begin{axis}[
scale = 1.2,
xmin = 0, xmax = 10,
ymin = 0, ymax = 10,
axis lines* = left,
xtick = {0}, ytick = \empty,
clip = false,
]
% Colouring areas
\fill[teal, opacity = 0.1] (0, 3.53) -- (4.14, 3.53) -- (4.14, 5.96) -- (0, 5.96);
\fill[violet, opacity = 0.1] (0, 3.53) -- (4.14, 3.53) -- (4.14, 3) -- (0, 3);
\fill[teal, opacity = 0.25] (4.14, 3.53)-- (5.05, 3.53) -- (4.14, 5.96);
\fill[violet, opacity = 0.25] (4.14, 3)-- (5.05, 3.53) -- (4.14, 3.53);

% Supply and demand curves
\addplot[color = blue, very thick] coordinates {(3,9) (6,1)};
\addplot[color = red, very thick] coordinates {(1,1) (9,6)};
\addplot[color = red, opacity = 0.3, very thick] coordinates {(1,4) (9,9)};

% Dashed lines
\addplot[color = black, dashed, thick] coordinates {(0, 5.96) (4.14, 5.96) (4.14, 0)};
\addplot[color = black, dashed, thick] coordinates {(0, 3.53) (5.05, 3.53) (5.05, 0)};
\addplot[color = black, dashed, thick] coordinates {(0, 3) (4.1, 3)};

% Coordinate points
\addplot[color = black, mark = *, only marks, mark size = 3pt] coordinates {(4.14, 5.96) (5.05, 3.53)};

% Labels
\node [right] at (current axis.right of origin){Gasoline, $Q$};
\node [above] at (current axis.above origin) {Price, $P$};
\node [above] at (5.25, 3.6) {$E$};
\node [above] at (4.35, 6.1) {$E^\prime$};
\node [right] at (6, 1) {$D$};
\node [right] at (9, 6) {$S$};
\node [right] at (9, 9) {$S^\prime$};
\node [left] at (0, 3.7) {$P_E$};
\node [left] at (0, 2.8) {$P_S$};
\node [left] at (0, 5.96) {$P^\prime$};
\node [below] at (5.05, 0) {$Q_E$};
\node [below] at (4.14, 0) {$Q^\prime$};
\node [above] at (2, 4.7) {$A$};
\node [above] at (1, 2) {$B$};
\node [above] at (4.5, 3.5) {$C$};
\node [above] at (6.2, 2.2) {$F$};

% Arrows
\draw[-{Triangle[length=4mm, width=2mm]}, red, opacity = 0.3] (8, 5.8) to (8, 7.8);
\draw[-Triangle] (1.3, 2.4) to [out = 0, in = 270] (2, 3.35);
\draw[-Triangle] (5.9, 2.6) to [out = 180, in = 270] (4.3, 3.4);

% Dimension lines
\draw[|-|] (4.14, -1) to (5.05, -1);
\node [below] at (4.59, -1) {Consumption reduction};
\draw[|-|] (-1, 5.96) to (-1, 3.8);
\node [below, align = left] at (-2.3, 6) {Consumer \\ incidence};
\draw[|-|] (-1, 3.6) to (-1, 2.8);
\node [below, align = left] at (-2.3, 3.5) {Producer \\ incidence};
\draw[|-|] (-3.6, 5.96) to (-3.6, 2.8);
\node [below, align = left] at (-4.2, 4.7) {Tax};

% Legend
\node [below right, draw, align = left] at (10.5, 10) {
\fcolorbox{black}{teal!10}{\makebox[\fontcharht\font`X]{$A$}} + \fcolorbox{black}{teal!25}{\makebox[\fontcharht\font`X]{$C$}} : Consumer  \\
\fcolorbox{black}{violet!10}{\makebox[\fontcharht\font`X]{$B$}} + \fcolorbox{black}{violet!25}{\makebox[\fontcharht\font`X]{$F$}} : Producer \\
\fcolorbox{black}{teal!10}{\makebox[\fontcharht\font`X]{$A$}} + \fcolorbox{black}{violet!10}{\makebox[\fontcharht\font`X]{$B$}} : Total  \\
\fcolorbox{black}{teal!25}{\makebox[\fontcharht\font`X]{$C$}} + \fcolorbox{black}{violet!25}{\makebox[\fontcharht\font`X]{$F$}} : Deadweight loss
};
\end{axis}
\end{tikzpicture}

\subsection{Polares}
\begin{figure}[h!]
    \centering
    \begin{tikzpicture}[scale=0.7]
        \begin{polaraxis}
            \addplot[
            red,
            thick,
            samples=600,
            domain=0:360
            ]{2*(1-sin(x))};
        \end{polaraxis}
    \end{tikzpicture}
    
    \caption{Polar $r=2(1-sin(\theta))$}
    \label{gra:polar}
\end{figure}

\subsection{Radial}

\bigskip
\begin{tikzpicture}[scale=0.5]
    \coordinate (origin) at (0, 0);

    \foreach[count=\i] \radius/\dim in {4/Proficiency,
                                        9/Growth,
                                        1/Ratio,
                                        7/Duration,
                                        3/Motivation,
                                        7/Breadth,
                                        2/Depth}{
        \coordinate (\i) at (\i * 360 / 7: \radius);
        \node (title) at (\i * 360 / 7: 11) {\Huge\dim};
        \draw (origin) -- (title);
    }

    \draw [fill=blue!20, opacity=.7] (1)
                                \foreach \i in {2,...,7}{-- (\i)} --cycle;
\end{tikzpicture}

\newpage
\subsection{Pie}
% \begin{figure}[h!]
%     \centering
%     \begin{tikzpicture}[scale=0.8]
%         \pie[text=legend,style=drop shadow,explode={0,0.5,0,0}]{20/Samsung,21/Huawei,12/Apple,8/Xiaomi,40/otros}
%     \end{tikzpicture}
%     \caption{Pie}
%     \label{gra:pie}
% \end{figure}

\begin{figure}[h!]
    \centering
    \begin{tikzpicture}
        \pie[pos={0,0},radius=1]{10/,20/,30/,40/}
        \pie[pos={2.7,0},radius=1.5]{10/,20/,30/,40/}
        \pie[pos={6.5,0},radius=2]{10/,20/,30/,40/}
    \end{tikzpicture}
    \caption{Multi Pie}
    \label{gra:mpie}
\end{figure}


\newpage
\subsection{Diagrama de Barras}
\begin{figure}[h!]
    \centering
    \begin{tikzpicture}[scale=0.8]
        \begin{axis}[
        ybar,
        x tick label style={/pgf/number format/1000 sep=},
        enlarge x limits = 0.2,
        width=12cm,
        height=7cm,
        ylabel = Frecuencia,
        legend style={at={(0.5,-0.15)},anchor=north,legend columns=-1}
        ]
            \addplot[azul,fill=azul] plot coordinates {(2014,5)(2015,3)(2016,2)(2017,1)(2018,1)(2019,2)};
            \addplot[rojo,fill=rojo] plot coordinates {(2014,2)(2015,5)(2016,8)(2017,1)(2018,4)(2019,2)};
            \addplot[verde,fill=verde] plot coordinates {(2014,4)(2015,2)(2016,4)(2017,5)(2018,6)(2019,5)};
            \legend{Categoría 1,Categoría 2,Categoría 3}
        \end{axis}
    \end{tikzpicture}
    \caption{Diagrama de Barras}
    \label{gra:Barras}
\end{figure}

\bigskip
% \begin{tikzpicture}[scale=0.75]
% % relleno para la cuadrícula de color amarillo
% \draw[yellow!10, fill = yellow!10] (0,0) rectangle (7,8); 
% \draw[red,dotted] (0,0) grid (7,8); 
% % Primeros escalones normales
% \draw[blue,thick,fill=blue!25] (0,0) rectangle (1,1);
% \draw[blue,thick,fill=blue!25] (1,0) rectangle (2,2);
% \draw[blue,thick,fill=blue!25] (2,0) rectangle (3,3);
% \draw[blue,thick,fill=blue!25] (3,0) rectangle (4,4);
% \draw[blue,thick,fill=blue!25] (4,0) rectangle (5,5);
% \draw[blue,thick,fill=blue!25] (5,0) rectangle (6,6);
% \draw[blue,thick,fill=blue!25] (6,0) rectangle (7,7); 
% \draw [->](3,-0.35) -- (0,-0.35);
% \draw [->](4,-0.35) --(7,-0.35);
% \draw(-0.15,8) --(-0.55,8); % Para indicar el límite
% \draw(-0.15,1) --(-0.55,1); % Para indicar el límite
% \node[below] at (3.5,0) {$n$};
% \draw [->](-0.35,4) -- (-0.35,1);
% \draw [->](-0.35,5) --(-0.35,8);
% \draw(7,-0.15) --(7,-0.55); % Para indicar el límite
% \draw(0,-0.15) --(0,-0.55); % Para indicar el límite
% \node[left] at (0,4.5) {$n$};
% %
% \draw [->](7.8,3.5) -- (7.8,0);
% \draw [->](7.8,4.5) --(7.8,8);
% \node[right] at (7,4) {$n+1$};
% \draw(7.25,8) --(8.35,8); % Para indicar el límite
% \draw(7.25,0) --(8.35,0); % Para indicar el límite
% %
% \node at (0.5,0.5) {1};
% \node at (1.5,1) {2};
% \node at (2.5,1.5) {3};
% \node at (3.5,2) {4};
% \node at (4.5,2.5) {$\cdots$};
% \node at (6.5,3.5) {$n$};
% \end{tikzpicture}
	
\end{document}
